\chapter{Laser speckle perfusion imaging reveals large-scale synchronization of cortical autoregulation dynamics influenced by nitric oxide}
\chaptermark{Synchronization of renal cortical perfusion}

\small{\textit{(Nicholas Mitrou, Christopher G Scully, Branko Braam, Ki H Chon, William A Cupples. American Journal of Physiology: Renal Physiology. XXX: Fxxx-Fxxx. 2014)}}

\section{Introduction}

Tubuloglomerular feedback (TGF) and the myogenic response (MR) are the canonical
mechanisms that mediate autoregulation of renal blood flow (RBF). Both mechanisms control the
diameter of the afferent arteriole to change preglomerular vascular resistance and both contribute to the gain of autoregulation \cite{Hayashi89,Just98,Persson82}. This process is often considered in the context of each nephron and its afferent arteriole responding individually to changes in blood pressure (BP) \cite{GuytonHall}. This view of autoregulation is being superseded by the increasing recognition that the terminal microcirculation in many beds operates as a set of distributed networks \cite{Krupatkin10}. In such a network the entire network contributes to determining flow in any one vessel; restated this says that regulation of flow requires coordination of resistance changes along multiple segments in any given pathway \cite{Mahmud13}.

	In kidney there is abundant evidence that TGF is synchronized in nephron pairs and triplets supplied by the same cortical radial artery \cite{HolsteinRathlou87}, that this synchronization could contribute to the efficacy of autoregulation \cite{Kallskog90}, and that hypertensive models  display altered synchronization of TGF in neighbouring nephrons \cite{Yip92,Chen95}. The MR also shows synchronized dynamics in nephron pairs and triplets \cite{Sosnovtseva07}. Both TGF \cite{Wagner97} and the MR \cite{Rivers95} are vascular conducted responses whereby constriction or dilation is conducted along the endothelium of a blood vessel with length constants $\approx 300 \mu$m \cite{Wagner97}, or $2 \ \times$ the length of the afferent arteriole \cite{Casellas94}. These findings strongly suggest that synchronization of TGF involves, at a minimum, all 15-20 nephrons in a lobule. However, there is little evidence in the literature of synchronization of TGF or MR at the scale of the lobule or larger. Scrutiny of the relevance of synchronization to autoregulation has been hampered by the inability to study more than two or three nephrons simultaneously. While there are data in the literature that are consistent with wide scale synchronization of MR, those data by no means demonstrate the phenomenon \cite{Cupples96}.
	
	Recently laser speckle contrast imaging (LSCI) has been shown to capture the dynamics of renal autoregulation at the renal surface \cite{HolsteinRathlou11,Scully13,Scully14}. Holstein-Rathlou \emph{et al.} \cite{HolsteinRathlou11} identified strong pairwise synchronization of TGF dynamics by hierarchical clustering. Figures 4, 6, and 7 in that report suggest the possibility of synchronization. Scully \emph{et al.} showed that renal surface perfusion can be segmented into clusters based upon synchronized MR dynamics and that these clusters were surprisingly large \cite{Scully14}.
	
	The importance of addressing MR dynamics is illustrated by studies with the Fawn hooded hypertensive rat and recombinant inbred strains derived from it. While TGF is intact in this strain \cite{Verseput98}, the MR is conspicuously absent, resulting in gross impairment of autoregulation and rapid development of renal failure \cite{vanDokkum99a,vanDokkum99b,vanRodijnen02,Ochodnicky10}. Notably, a quantitative trait locus on rat chromosome 1 accounts for most of the impairment of autoregulation \cite{Lopez06} and the bulk of the renal failure \cite{Brown96}. These studies provide strong evidence that the MR is both necessary and sufficient for effective autoregluation (reviewed in \cite{Cupples07a,Cupples07b}).
	
	Here we report studies designed to assess synchronization of TGF and MR over larger scales than previously - from the lobule ($\approx 0.6 \ $mm diameter) to the field of view ($\approx 4 \times 5 \ $mm) or 50 - 60 lobules. Synchronized regions of the kidney surface were identified by phase coherence between all possible pairs of pixels from the imaged surface ($27 \ \times \ 37$ pixels). $N_\omega$-nitro-L-arginine methyl ester (L-NAME) was used to increase renal vascular tone and change autoregulation dynamics at afferent \cite{Wang99,Wang01} and efferent arterioles \cite{Sosnovtseva09}, actions that are largely in dependent of BP \cite{Shi06,Wang99,Wang01,Wang07}. The results show widespread moderate (TGF) or modest (MR) synchronization that was substantially and differentially altered by inhibition of nitric oxide synthases.


\section{Methods}

	All experiments were approved by the Animal Care Committee of Simon Fraser University and performed according to the guidelines of the Canadian Council on Animal Care. Male Long-Evans rats (aged 12-14 weeks, $N=7$) were purchased from Charles River Canada, Ltd. (Ste. Hyacinthe, QC,Canada) or Harlan Laboratories (Livermore, CA, USA), house in groups of 3-5 with \emph{ad libitum} access to standard rat chow and distilled water.
	
	Each rat received the analgesic buprenorphine (Summit Veterenary Pharmacy, Aurora, ON, Canada; 0.02 mg/kg s.c.) 20 minutes prior to anesthesia. Anesthesia was induced with 4\% isoflurane (Aerrane; Baxter, Mississauga, ON, Canada) in inspired air (45\% O\textsubscript{2}). After induction the rat was transferred to a heated surgical table ($35 ^{\circ} \ {\rm C}$). The trachea was cannulated (PE-210) and the rat was ventilated with a pressure-controlled small animal ventilator (TOPO, Kent Scientific, Torrington, CT, USA) operating in times ventilation mode and adjusted to match the rat's natural breathing rat (50-60 min \textsuperscript{-1}). During surgery inspired isoflurane was set to 2\%.
	
	The left femoral vein was cannulated (PE-50) for infusion of 2\% charcoal-washed bovine serum albumin (Sigma-Aldrich, Oakville, ON, Canada) in normal saline at 1\% BW/h. The left femoral artery was cannulated (PE-90 with narrowed tip) and connected to a pressure transducer (TRN050, Kent) driven by a TRN005 amplifier for measurement of BP. The left kidney was exposed by a subcostal flank incision, freed from surrounding fat, and placed in a plastic cup mounted to the surgical table. The kidney was bedded in silicone stopcock grease (Dow Corning, Midland, MI, USA) to minimize motion. After the renal artery was stripped a transit time ultrasound flow probe (PRB-001, Transonic Systems, Ithaca, NY, USA) was attached. The flow probe was secured in place with acoustic coupling gel (NALCO 1181 and surgical lubricant) and driven by a TS-420 flow meter (Transonic). Except when acquiring data, the surface of the kidney was covered with Parafilm to prevent drying. Anesthesia was reduced to the lowest dose ($\approx 1.5\%$) that prevented BP responses to toe pinching and the animal was allowed to stabilize for one hour.
	
	The kidney was prepared by gently cleaning the surface with normal saline and periodically moistening it thereafter. LSCI was performed with a moorFLPI laser speckle contrast imager (Moor Instruments, Axminster, UK) positioned $\approx 18$ \ cm above the kidney surface. The flattest portion of the dorsal surface of the kidney was chosen for imaging. Approximately one third of the total dorsal surface was monitored. The instrument was modified by the manufacturer to have twice the optical zoom of the standard model allowing a small field of view $\approx 4\times 5$ mm (Table 1).
	
	To avoid aliasing heart rate-dependent flow pulses into perfusion \cite{Scully13}, the imager was used in spatial averaging mode which acquires $113 \times 152$ pixel images of perfusion at $25$ \ Hz. Before each record was taken, a $4$ \ mm hair was placed on the surface to determine pixel size and to assist in focusing the LSCI camera. 
	
	\emph{Experiments}. Previous studies showed that record lengths of $30-50$ \ cycles ($300$ \ s records) are required to probe MR synchronization \cite{Scully14} and pilot studies showed the same $30-50$ \ cycle ($1500$ \ s) requirement for probing TGF synchronization. BP, RBF, and LSCI signals were acquired during conditions of spontaneous BP fluctuation before (control) and after i.v. injection of L-NAME ($10$ \ mg/kg) in $10$ rats. Three animals were removed from analysis due to lack of L-NAME dependent vasoconstriction ($N = 1$) or lack of a detectable myogenic dynamic in RBF ($N=2$), leaving $7$ animals for full analysis. 
	
	\emph{Synchronization and cluster analysis}. Detection of synchronized regions on the surface of the kidney used methods described in detail elsewhere \cite{Scully14} that are based on assessment of the stability in time of phase relationships among pixels. Breifly, each image was spatially filtered with a Gaussian spatial filter ($8$ \ pixel width) and then down sampled by a factor of $4$, yielding a $27\times 37$ \ pixel image for each frame with average pixel size $135\times 135 \ \mu \ $m \cite{Scully14,Scully13}. The TGF ($0.015-0.06 \ $ Hz) and MR ($0.09-0.3 \ $Hz) frequency ranges were isolated by a bandpass forwards-backwards Butterworth filter so that phase information was preserved. Then the Hilbert transform of the data was used to allow the extraction of instantaneous phase. The Hilbert transform of a real number produces a complex number, the arc tangent of which is the instantaneous phase of the oscillation. Synchronization was assessed by calculating the phase coherence (PC, eq. 2.1) between all possible pairs of imaged pixels.

\begin{equation}	
PC_{jk} = \lvert \frac{1}{N} \sum \limits_{m=1}^{N} e ^{i(\phi_j - \phi_k)} \rvert
\end{equation}

	$PC_{jk}$ \ is the phase coherence between pixels $j$ and $k$ where $N$ is the number of points in the original data, and $\phi _{j}$ and $\phi _{k}$ are the instantaneous phases of pixels $j$ and $k$ \ , respectively. It is the mean of the exponential difference between the instantaneous phases of two pixels over time. PC is bounded by $0$ and $1$ where $PC=0$ means there is no phase relationship between the two pixels and $PC=1$ indicates complete $1:1$ phase locking.

	A surrogate data approach was used to provide an estimate of PC that could be considered significant. For each pixel pair, PC was determined after a random phase circular phase shift was applied to one pixel. The significance threshold was set to two standard deviations greater than the mean PC from 50 iterations of this procedure. Pixel pairs that had PC less than the threshold were set to 'not a number' and therefore excluded from further analyses. PC for TGF was computed over the entire $1500 \ $ s record while PC for MR was computed over $300 \ $ s segments with $150 \ $ s overlap. We did this because the threshold for significance decreases as the number of cycles increases in a data set, so a similar number of cycles was used to analyze MR and TGF \cite{Scully14}.
	
	Calculating pairwise PC allows comparison of any particular, or reference, pixel to all the other pixels in the field of view. Simply moving the reference pixel can give some indication of the presence and location of synchronized regions, but would be cumbersome and insensitive. Instead, we used cluster analysis of the PC values to determine the number of synchronized clusters and their location on the renal surface. From the available methods of clustering, we tested three and chose to use non-negative matrix factorization (NMF) because it was the most accurate in the presence of high PC between clusters \cite{Scully14}. We have described this procedure in detail \cite{Scully14}. Briefly, the pairwise PC results were first transformed from a four-dimensional $27 \times 37 \times 27 \times 37 \ $ matrix to a $999 \times 999 \ $ synchronization matrix that held the PC values between all pairs of pixels. Then, we used NMF to estimate the number and location of synchronized clusters. NMF is an algorithm designed to approximate the $n \times n \ $ matrix $V$ into two vectors, $W \ (n \times r) \ $ and $W \ (n \times r) \ $, where $r$ is the number of clusters, so that equation 2.2 is satisfied.
	
\begin{equation}
WH \approx V
\end{equation}

	The NMF calculation was iterated with increasing values of $r$, from $1$ to $10$, and the resulting error was computed after each iteration according to equation 2.3. 

\begin{equation}
D = \lvert \lvert WH - V \rvert \rvert ^2
\end{equation}

	At the first error reduction less than $0.01$ after increasing $r$ by $1$, \emph{i.e.} $D_{r+1} < 0.01$, the number of clusters was set to $r$. For the chosen number of clusters, $r$, the matrices $W$ and $H$ are used to determine the cluster that each pixel belongs to by finding the cluster $m$ that produces the largest contribution to the estimate of $V$ according to equation 2.4.

\begin{equation}
Cluster(m) = arg \ max_m(W_{nm}H_{mn})
\end{equation}

	Cluster dimensions were estimated from the pixel dimensions in individual experiments and the number of pixels contained within each cluster. For this purpose, edges of the field of view were treated as cluster boundaries. Thus the estimates of cluster area must be considered as estimates at best and as lower limits particularly when the number of clusters in the field is small.

	PC between the geometric centroid of each cluster and all other significant pixels in the same cluster was used to provide an index of the overall strength of synchronization in each cluster. The absolute value of the phase difference (PD) between two pixels provides an indication of the coupling mechanism. Low PD, near $0$ radians (rad), suggests electrotonic coupling whereas high PD, near $\pi$ rad, suggests hemodynamic coupling. To assist in interpretation we assessed PC and PD using test signals: identical, in-phase, sine waves. PC and PD were calculated between the two waves for 100 repetitions. Then Gaussian white noise was added to one of the waves in 120 increments, so that signal to noise ratio (SNR) ranged from $-20$ to $40$ dB, and PC and PD were calculated at each level of noise. Phase for each signal was estimated using the methods described above. The process was repeated using a wavelet-based method, as reported in \cite{HolsteinRathlou11}. Briefly, the wavelet-derived time frequency spectrum was used to determine the dominant frequency of the signal. Then the phase was taken at the dominant frequency at each time point. 
	
	TGF clusters after L-NAME showed pronounced radial heterogeneity of PC and PD within clusters. We systematically moved the PC reference pixel across the PC image to test whether high PC regions within clusters were determined by the location of the reference pixel or whether multiple reference pixel locations captured the same high PC regions.
	
	\emph{Statistical analysis}. Differences between experimental periods and between MR and TGF were assessed with paired t-tests. Because PC values are bounded from $0$ to $1$ and thus not normally distributed, differences were tested with the Wilcoxon signed-ranks test. Data acquisition and analysis was performed in Matlab (r2013a, The Mathworks, Natick, MA, USA) and statistical tests were performed in SPSS (v22, IBM, Armonk, NY, USA). Data are shown as mean $\pm$ standard error and $P<0.05$ was considered statistically significant.	

\section{Results}

	During the control period the rats had BP $ = 101 \pm 5 \ $ mmHg, RBF $ = 7.0 \pm 0.7 \ $ mL/min, and LSCI $ = 1620 \pm 35 \ $ AU. After L-NAME, BP increased to $133 \pm 3 \ $ mmHg, RBF decreased to $4.4 \pm 0.5 \ $ mL/min, and LSCI decreased to $1000 \pm 104 \ $ AU; $P<0.01 \ $ in all cases.
	
	Figure 2.1 shows a basic representation of the signals that were extracted from each LSCI image series. Panel A shows the flux across the renal surface averaged over 25 minutes. The mean flux is not homogeneous across the surface and instead has a region of high flux at the top right that transitions to low flux in the bottom left. Panel B shows 300 s segments of filtered flux time series extracted from three single-pixel regions in interest (ROI, $192 \mu m^2$), chosen so that two are neighbours (separated by $271 \mu m$) and the third is remote from that pair. The waveforms of the two neighbouring ROI appear similar while that of the remote ROI is visibly dissimilar. Time-varying spectral power was estimated by the wavelet transform of each ROI \cite{Scully13}. Time-varying spectra for neighbouring ROI (panels C and D) show largely synchronous power fluctuation in both MR and TGF bands. In contrast, the time-varying spectrum of the remote ROI (panel E) shows clear temporal dissociation of power in MR and TGF. Panel 2.1F shows that the neighbouring ROI have near zero phase difference, suggesting a high degree of phase coherence, whereas distant ROI have rapidly fluctuating phase difference, suggesting no synchronization. PC between the two nearby ROI (PC\textsubscript{1,2}) was 0.81 while PC\textsubscript{1,3} was 0.33 and PC\textsubscript{2,3} was 0.28. Although inspection of PD\textsubscript{1,2} shows it is close to zero, phase slips ($\pm 2\pi$) introduce noise so PD\textsubscript{1,2} $> 0\ $ and PC\textsubscript{1,2} $< 1\ $. 
	
\begin{figure}[H]
\begin{center}
\includegraphics[scale = 0.8]{Figures/Chapter02/Fig1_12B23_hot_Arial.eps}
\caption[Summary of cortical perfusion dynamics]{\textbf{A}: Laser speckle flux on a $5\times 7 \ $ mm area of the renal surface averaged over 1500 seconds. Three regions of interest (ROI) are labeled 1-3. The color of each label corresponds to the color of subsequent time series. \textbf{B}: Time series of flux extracted from each ROI. 300 seconds are shown. Fluctuations in flux are similar between ROI 1 (\textbf{black}) and 2 (\textbf{red}) but not between ROI 1 or ROI 2 and ROI 3 (\textbf{blue}). \textbf{C}: Time-varying power spectrum from ROI 1. There are intermittent contributions of power from TGF and MR, and these changes in power are similar and temporally aligned to equivalent events in the power spectrum from ROI 2 (\textbf{D}). \textbf{E}: Power spectrum from ROI 3 shows fluctuations in power and contains information from both MR and TGF. However the events that occur in ROI 3 are not similar in power or time to ROI 1 or ROI 2. \textbf{F}: Difference in instantaneous phase in the myogenic range ($0.09 - 0.3 \ $ Hz) between ROI 1 and 2 (black) and between ROI 1 and 3 (red). Phase differences were constrained to $[-\pi , \pi]$. The small phase difference between the neighbouring ROI is consistent with their synchronization whereas the large phase differences between distant ROI}
\end{center}
\end{figure}

To estimate the robustness of these indices, we determined PC and PD for test signals under different conditions. Results of this test are shown in figure 2.2. As expected in-phase sine waves have PC $= 1 \ $ and PD $ = 0 \ $ when SNR is very high. The response to added noise was sigmoidal was SNR was reduced from 40 dB to -20 dB. For both the Hilbert and wavelet methods, added noise caused PC to decrease from 1 until it approached zero at very low SNR, while PD increased from 0 to approximately $\frac{2}{3}\pi$. However, SNR in the region from $[-10,15 \ dB]$, the Hilbert transform detected PC $> 0 \ $ that the wavelet method did not. This SNR range also corresponded to the values of PC that we observed in our rat data. Because of the way the wavelet transform was used, low SNR caused the dominant frequency to be masked by other frequencies that had high spectral power solely because of noise. For this reason, the wavelet technique tended to break down at low SNR while the Hilbert technique did not.

\begin{figure}[H]
\begin{center}
\includegraphics[scale = 1]{Figures/Chapter02/Sim_Hilbert_VS_Wavelet_PCPD_Single_errorbar_piLabels.eps}
\caption[Comparison of phase coherence calculated with Hilbert and wavelet methods]{\textbf{A}: Phase coherence (PC) between two identical, in phase, sine waves, computed from instantaneous phase estimated with the Hilbert transform \textbf{black} or the wavelet transform \textbf{grey}. PC was calculated between the two waves as Gaussian white noise was added to one of the waves at 60 levels of signal to noise ratio (SNR) ranging from -20 dB to 40 dB. This process was repeated 100 times for each level of SNR. \textbf{B}: Phase difference (PD) between the same sine waves as in \textbf{A} calculated in the same way. For both techniques, PC was 1 and PD was 0 when SNR was high. As SNR was reduced, PC decreased sigmoidally until it approached 0, while PD increased sigmoidally until it approached $\frac{2}{3}\pi$. As SNR became lower than 15 dB, the wavelet technique failed to detect PC more quickly than the Hilbert technique did, suggesting that the Hilbert transform method is more robust in the presence of low SNR than the wavelet method.}
\end{center}
\end{figure}

	Figure 2.3 shows synchronization of MR in one rat before and after L-NAME. During the control period, portions of two clusters are visible in the field of view. PC in each cluster was modest, with mean PC $= 0.50$, and PD $= 0.4\pi \ $ rad; both PC and PD were homogeneously distributed. The coefficient of variation of PC (CV\textsubscript{PC}) was 10\%, consistent with the even distribution of PC across the field of view. After L-NAME, the number of clusters increased from two to five and they had a greater PC and lower PD than during the control period. Mean within-cluster PC was 0.67 in this animal while the within-cluster PD was $0.3\pi \ $ rad. The edges of each cluster retained lower PC and higher PD, making the cluster boundaries more clearly demarcated. Mean within-cluster PC was altered so that the CV\textsubscript{PC} increased to 19\%.
	
\begin{figure}[H]
\begin{center}
\includegraphics{Figures/Chapter02/Fig3_12B23_myo_clusters_vert_hot_Arial.eps}
\caption[Myogenic synchronization before and after L-NAME]{Myogenic synchronization during control (Left) and after L-NAME (Right). The number of synchronized clusters in the field of view is shown in the top row. The black bar in all images masks a 4 mm scaling hair. Phase coherence between all the pixels in each cluster and the centroid of each cluster is shown in the middle row. The bottom row shows mean phase difference between the centroid of each cluster and the remaining pixels in each cluster.}
\end{center}
\end{figure}

	MR synchronization in all seven rats showed similar patters to the rat shown in figure 2.3 and group averages are reported in Table 2.1. During the control period, mean within-cluster PC was modest and mean PD was $<\frac{\pi}{2} \ $ rad. All animals showed evenly distributed patterns of PC and PD, reflected by low CV\textsubscript{PC} and CV\textsubscript{PD}. After L-NAME the average number of clusters was significantly increased so that estimated area was significantly and substantially reduced. Mean PC increased and mean PD decreased while CV\textsubscript{PC} and CV\textsubscript{PD} increased. These results are consistent with widespread and homogeneous, though weak, synchronization of MR during control and stronger, more localized synchronization after L-NAME.

%Table 2.1 should go here
\begin{table}[h]
\begin{center}
	\caption[Summary of synchronization during Control and L-NAME]{\emph{Internephron synchronization of MR and TGF before and after L-NAME}}
	\begin{tabular}{c c c c}
		 &         &          & \\
 		 & SPN & L-NAME & \emph{P}-value \\
		 &         &          & \\
		 &         & \emph{Field of View} & \\
		 &         &          & \\
 		X Dimension & $5.2 \pm 0.3$ & $5.4 \pm 0.3$ & \\ 
		Y Dimension & $3.8 \pm 0.2$ & $3.9 \pm 0.2$ & \\
		 &         &          & \\
		 &         & \emph{MR} & \\
		 &         &          & \\
		 No. of Clusters & $1.3 \pm 0.2$ & $4.4 \pm 0.5$ & 0.001 \\
		Estimated Area & $17 \pm 2$ & $5.1 \pm 0.7$ & 0.001 \\
		PC & $0.4 \pm 0.1$ & $0.6 \pm 0.1$ & 0.014 \\
		PD & $1.4 \pm 0.1$ & $1.2 \pm 0.1$ & 0.041 \\
		CV\textsubscript{PC} & $12 \pm 1$ & $21 \pm 2$ & 0.001 \\
		CV\textsubscript{PD} & $5 \pm 1$ & $20 \pm 3$ & 0.003 \\
				 &         &          & \\
		 &         & \emph{TGF} & \\
		 &         &          & \\
		 No. of Clusters & $1.9 \pm 0.1$ & $2.9 \pm 0.4$ & 0.038 \\
		Estimated Area & $11 \pm 1$ & $8 \pm 1$ & 0.039 \\
		PC & $0.7 \pm 0.1$ & $0.5 \pm 0.1$ & 0.016 \\
		PD & $1.0 \pm 0.1$ & $1.3 \pm 0.1$ & 0.022 \\
		CV\textsubscript{PC} & $11 \pm 2$ & $24 \pm 4$ & 0.048 \\
		CV\textsubscript{PD} & $13 \pm 2$ & $17 \pm 2$ & 0.087 \\
	\end{tabular}
		
\end{center}
\end{table}
	
	Figure 2.4 shows TGF synchronization, before and after L-NAME, from the same rat as figure 2.3. During the control period, portions of two clusters are visible; mean PC was 0.72 and mean PD was $0.3\pi \ $ rad. Other than the magnitude of these values, the visual appearance of PC and PD was similar to MR during the control period. Low CV\textsubscript{PC} (8\%) and CV\textsubscript{PD} (13\%) were consistent with the even appearance of clusters during the control period. After L-NAME, three clusters were visible in the field of view. Within-cluster PC decreased to 0.43 and within-cluster PD increased to $0.47\pi \ $ rad. CV\textsubscript{PC} increased to 39\%, while CV\textsubscript{PD} increased to 20\%, confirming the increased heterogeneity of each cluster and the appearance of venter-surround-like organization within each cluster.
	
\begin{figure}[H]
\begin{center}
\includegraphics[scale=0.6]{Figures/Chapter02/Fig4_12B23_tgf_clusters_vert_hot_Arial_PDFprint}
\caption[Tubuloglomerular feedback synchronization before and after L-NAME]{Tubuloglomerular feedback synchronization before (Left) and after (Right) L-NAME. The location of synchronized clusters is shown in the top row. The black bar masks a 4 mm scaling hair. Phase coherence between all the pixels in each cluster and the centroid of each cluster is shown in the middle row. The bottom row shows mean phase difference between the centroid of each cluster and the remaining pixels in each cluster.}
\end{center}
\end{figure}

	TGF synchronization in all seven rats was similar to the animal shown in figure 2.4 and group averages are presented in Table 2.1. During the control period, mean within-cluster PC was moderate and mean PD was $0.3\pi \ $ rad with homogeneous PC and PD, as shown by low CV\textsubscript{PC} and CV\textsubscript{PD}. L-NAME caused the number of synchronized clusters to increase while the average within-cluster PC decreased and PD increased. Mean within-cluster PC and PD changed in the opposite direction compared to MR, reflecting the venter-surround appearance of TGF clusters after L-NAME.
	
	During the control period TGF clusters were more numerous than MR clusters ($P=6.4\times 10^{-6}$) and were consequently smaller (Table 2.1). PC was higher in TGF clusters than in MR clusters ($P=1.8\times 10^{-11}$). After L-NAME the number of TGF clusters was modestly though significantly increased so that the umber of TGF and MR clusters was not different ($P=0.062$) while PC was lower in TGF clusters ($P=0.001$). 
	
	Inspection of figures 2.3 and 2.4 reveals several important similarities between MR and TGF synchronization during control. In both cases the small peak in PC (and minimum of PD) is artifactual and due to correlation of the reference pixel with itself. Apart from this point, the rest of each cluster appears homogeneous with respect to PC and PD. After L-NAME, the MR clusters show a simple pattern with a broad region having high PC and low PD surrounded by a region at the cluster boundaries of low PC and high PD. In contrast, TGF clusters after L-NAME show a complex center-surround pattern in which the high PC, low PD centre is $\approx 4\times 4 \ $ pixels or $540\times 540 \  \mu m$, larger than the artifactual centres during the control period. This pattern may indicate differential synchronization of smaller regions within a synchronized cluster. 
	
	To explore the possibility of differential synchronization within a cluster, we systematically moved the reference pixel across the image of MR clusters (figure 2.5) and TGF clusters (figure 2.6), both after L-NAME. In both cases only pixels having significant PC with the reference pixel are shown. In figure 2.5, it is clear that moving the reference pixel within MR clusters has little effect (B versus C and D versus E) whereas a move of similar size that crosses a cluster boundary (C versus D) results in detection of different clusters. 
\begin{figure}[H]
\begin{center}
\includegraphics[scale = 0.7]{Figures/Chapter02/Fig5_12B23_MR_PC_move_SigOnly_Hot_Arial.eps}
\caption[Effect of reference pixel location on myogenic phase coherence]{Phase coherence (PC) in MR signals as the reference pixel is moved. Panel \textbf{A} shows PC after L-NAME from Figure 2 and placement of the reference pixels. Phase coherence is shown in panels \textbf{B}-\textbf{E} relative to a pixel identified in the top left panel. The reference pixel in panels \textbf{B} and \textbf{C} is in the same cluster and the region of high PC remains similar between the two pixel locations. When the reference pixel is moved to a new cluster (\textbf{C},\textbf{D}) the high PC region shifts into the new cluster. Only pixels having significant PC with the reference pixel are shown.}
\end{center}
\end{figure}	
The situation for TGF is more complex as illustrated in figure 2.6.  In this case the reference pixel is moved only small distances within a single coherent cluster. In all cases (B-E), the "new" referece pixel identified small high PC regions that had mutual low PC with the geometric centroid. Panels B-E show that moving the reference pixel may have little effect on the high PC regions (panels B and C) or it may cause the high PC region to shift position (panels D and E).

\begin{figure}[H]
\begin{center}
\includegraphics[scale = 0.7]{Figures/Chapter02/Fig6_12B23_TGF_PC_move_SigOnly_Hot_Arial.eps}
\caption[Effect of reference pixel location on tubuloglomerular feedback phase coherence]{Same as figure 4 except for TGF. A region of high PC surrounds the reference pixel in \textbf{B}. As the reference pixel is moved upward, (Panel \textbf{C}) the location of the high PC region is unchanged. In panel \textbf{D} the reference pixel was moved further with the result that the high PC region became smaller, perhaps because the reference pixel is on a boundary between regions. In \textbf{E}, the reference pixel was moved again and the high PC region has moved and changed shape to point upward and rightward instead of downward and leftward as it began. Only pixels having significant PC with the reference pixel are shown.}
\end{center}
\end{figure}


\section{Discussion}

	Previously, synchronization of tubular pressure that can be blocked by furosemide has been shown in nephron pairs and triplets originating from the same cortical radial artery \cite{HolsteinRathlou87,Kallskog90,Yip92}. Micropuncture studies by Marsh and colleagues demonstrated synchronization of TGF \cite{HolsteinRathlou87,Yip92,Chen95} and MR \cite{Sosnovtseva09} dynamics in pairs of nephrons, but did not identify synchronization over larger scales. A limitation of micro puncture is that the potential number of oscillators being probed is very small and, for MR, heavily filtered. In contrast, LSCI acquires data from a large fraction of the cortical surface, permitting assessment of potential distributed function. The first LSCI study showed that synchronization of TGF dynamics in pairs and triplets of star vessels is very common \cite{HolsteinRathlou11}. Whether synchronization occurs on larger scales has not been examined, although data in the literature are consistent with large-scale synchronization of TGF \cite{HolsteinRathlou11} and MR \cite{Cupples96}. Therefore this study examined synchronization of autoregulation over wider areas of cortical surface.
	
	LSCI reports signals that are proportional to the flux of red blood cells moving just under the surface of the kidney \cite{Forrester04,Boas10}. The estimate of flux provided by LSCI varies tightly and linearly with other techniques for blood flow measurement including laser Doppler flowmetry \cite{Legrand11}, capillary microscopy \cite{Bezemer10}, and ultrasound flowmeters \cite{Scully13,HolsteinRathlou11}. In this study LSCI flux varied with RBF at $240 \pm 45 \ $ AU $\cdot$ (mL/min)\textsuperscript{-1}. After capturing surface perfusion, we then isolated MR and TGF dynamics to selectively probe each mechanism. 
	
	The simple qualitative analysis shown in figure 2.1 demonstrates local, but not remote, synchronization of MR dynamics. The nearby ROIs in this figure are separated by $271\mu m$\ and the PD between them is frequently zero, which illustrates synchronization of separated ROIs that is stable over time. This result is in excellent agreement with those of Marsh \emph{et al.} \cite{Marsh13} who simulated TGF dynamics in a 16-nephron vascular network model and predicted similar low phase differences in pairs of nearby nephrons. Both the prediction \cite{Marsh13} and the data (figure 2.1) suggest that synchronization of autoregulation occurs within some spatial limits, but the limits cannot be determined from either analysis.
	
	Observation of a system's free-running oscillations is not sufficient to detect synchronization because one cannot distinguish between synchronization and coincidentally similar frequencies \cite{Pikovsky01}. However, synchronization is clear when the frequency of one signal changes and the second changes with it. Consistent with previous data \cite{Zou02,HolsteinRathlou11,Chon08}, the frequencies of MR and TGF oscillations in this study change with time, illustrated in figure 2.1C-E. Consequently, the phase of each pixel also changes with time. Phase locking is the standard definition of synchronization \cite{Balanov08,Pikovsky01} and stability of PD between two signals indicates the degree of phase locking. While PC indicates the presence and degree of synchronization between two signals, PD is needed to determine the temporal relationship between them. Thus, it is necessary to consider PC and PD together. Under ideal conditions with strongly and uniformly coupled one would see PC = 1 and PD = constant. However biological variation and measurement limitations ensure that all our signals are noisy. Accordingly, the test shown in figure 2.2 was performed to assist in interpretation of experimental results. It shows that PC = 1 and PD = 0 is unlikely to be achieved when studying spontaneous dynamics. Importantly it shows that PC and PD are reliably identify synchronization, even at low SNR, indicating that the analysis is a sensitive detector of synchronization. We compared the resilience to noise of our method, which uses the Hilbert transform, to another method, using the wavelet transform. We found that although both methods detect PC and PD when SNR is relatively high, the Hilbert based technique continues to detect PC and PD to lower SNR than the wavelet based technique. Previous studies using this wavelet technique \cite{HolsteinRathlou11} did not identify widespread synchronization that we have. It is possible that the synchronization is weak, and thus has a low SNR, so previous attempts to find synchronization could not detect it. Assignment of pixels to clusters raided from assessment of pairwise PC with no prior spatial inputs. Clusters identified in this way were almost always contiguous with well-defined boundaries (\emph{e.g.} figures 2.3 and 2.4). A few interlocking (jigsaw puzzle) boundaries were seen and these probably reflect limits of resolution.
	
	Based on known TGF synchronization within lobules \cite{HolsteinRathlou87,Yip92,Chen95}, measured mechanical length constants \cite{Wagner97}, and vascular anatomy, \cite{Braus24,Beeuwkes75}, we proposed that the lobule would form the minimum synchronized unit of autoregulation. While qualitative description of cortical vascular anatomy is excellent, there is little quantitative information available in the literature about the dimensions of cortical lobules. We have chosen to use an estimate provided by Professor W. Kriz (personal communication) that the spacing between cortical radial arteries is between 0.5 and 0.7 mm and that tubular surface profiles are often quadrilateral. We thus estimate lobular surface area to be 0.3 - 0.5 mm\textsuperscript{2} and large enough, $\geq 4\times 4 \ $ pixels or $\approx 0.3 \ $ mm\textsuperscript{2}, to be detected reliably \cite{Scully14}. The notably homogeneous distribution of PC within MR and TGF clusters during the control period suggests that there is evenly distributed network behaviour in the renal cortex. PD between the geometric centroid of each cluster and the rest of its pixels was used to estimate whether coupling was electrotonically or hemodynamically mediated. Electrotonic coupling is mediated by transmission, presumably along the endothelial cable \cite{Diep05}, of hyper polarizing or depolarizing voltages. Transmission of these voltages through myoendothelial gap junctions then governs the state of vascular smooth muscle cells. Because signal transmission is rapid, PD will be small. Hemodynamic coupling results from reciprocal resistance changes in neighbouring downstream terminal vascular beds \cite{Moore94} and is characterized by anti-phase coupling. This process has been the subject of two modelling studies, both of which predict that anti-phase hemodynamic coupling dominates only when electronic coupling is removed \cite{Marsh13,Postnov12}. However, we found that MR and TGF were synchronized across large distances with relatively unchanging PC and PD. If electrotonic coupling is only local, we would expect to see a small region of low PD instead of the large regions seen in both MR and TGF under control conditions and in MR after L-NAME. While PD increases with distance from the reference pixel, the mean of the absolute value of PD was routinely between $\frac{\pi}{3} \ $ and  $\frac{\pi}{2} \ $ rad across the kidney surface, which is consistent only with electrotonic coupling.
	
	The main result is that under control conditions both MR and TGF are synchronized over much wider areas as shown by the estimated cluster areas reported in table 2.1. These findings indicate that large numbers of lobules can form a synchronized cluster, perhaps 40 being modestly synchronized within an MR cluster and up to 30 begin moderately synchronized within a TGF cluster. This implies that synchronization requires communication well upstream, into arcuate arteries that are generally considered to be conductance vessels \cite{Zamir87}. Although conducted mechanical responses have not been reported in arcuate arteries, conducted calcium responses have been shown \cite{Goligorsky95} and these vessels are myogenically active \cite{Kauser91}. While juxtamedullary nephrons have lower TGF frequencies \cite{HolsteinRathlou89} that could de-synchronize neighbouring lobules, the frequency of TGF is labile \cite{Zou02,Chon08} (and figure 2.1) and transit time through the loop of Henle can vary widely \cite{Steinhausen76,Stumpe70}. It is thus possible that the TGF frequencies in juxtamedullary nephrons could entrain to that of the short looped nephrons, perhaps in a 1:2 or 2:3 frequency ratio. L-NAME had pronounced and differential effects on synchronization of MR and TGF, consistent with stronger MR synchronization and restructuring of both MR and TGF synchronization. Cluster dimensions remained larger than lobules as shown in figures 2.3 and 2.4.
	
	During the control period the overall pattern of synchronization of MR and TGF was similar with large moderately (TGF) or modestly (MR) synchronized clusters. This pattern suggests extensive communication among segments in a distributed, loosely couple network. Such widespread synchronization has implications for dynamic range of autoregulation. Synchronization on this scale allows MR and TGF to engage more resistance than just the afferent arteriole, increasing the range of BP within which autoregulation can stabilize RBF and GFR. Such a network has on the order of 300-400 nephrons and would provide smoothing in space and time \cite{Tran12}.
	
	L-NAME caused substantial, and differential, changes in synchronization of MR and TGF. After L-NAME MR clusters revealed stronger synchronization that was restricted to smaller areas with monotonically distributed PC as shown in figures 2.3 and 2.5. The response of TGF synchronization to L-NAME was visibly different from that of MR. TGF clusters exhibited a venter-surround organization after L-NAME with only a central core exhibiting high PC and low PD. Systematically moving the reference pixel showed that for MR clusters the location of the reference pixel within a cluster is irrelevant to what the analysis reports (figure 2.5). In contrast, location of the reference pixel within a TGF cluster is relevant to what the analysis reports (figure 2.6). Thus it became apparent that individual TGF clusters are made up of smaller regions having high local PC and low PD that tend to be stable when the reference pixel is moved. These regions displayed mutual low PC and high PD with nearby regions in the same cluster. These regions were $\approx 5 \times 5 \ $ pixels, or $700\mu m \ $ on a side, similar to the expected dimensions of cortical lobules. Thus we interpret these synchronization patterns to capture strong synchronization in individual lobules with weaker, though still phase coherent, synchronization among neighbouring lobules in the same cluster. This may help reconcile LSCI results with earlier micropuncture studies of TGF synchronization. We believe that the high sensitivity in this study and in \cite{HolsteinRathlou11} enable one to detect both the strong synchronization of small regions that was previously detected by micro puncture and weaker, though still significant, synchronization of much larger areas.
	
\section{Conclusion}

	In summary, we have shown that both MR and TGF are extensively synchronized on macroscopic scales with synchronized clusters incorporating multiple lobules. We applied LSCI and a rigorous definition of synchronization to show that there is sustained synchronization of TGF and MR dynamics on the renal surface. Synchronziation of both mechanisms is modulated by L-NAME although the result of this modulation is different for MR and TGF. They show equally widespread synchronization of the MR and they show differential alterations of MR and TGF synchronization upon removal of nitric oxide.

\bibliographystyle{plainnat}
\renewcommand{\bibname}{References}
\bibliography{../Dissertation_Bib}











