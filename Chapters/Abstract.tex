%\documentclass{article}


%\begin{document}
\prefacesection{Abstract}
Renal autoregulation is the system that maintains a constant renal blood flow when renal perfusion pressure fluctuates. Failure of autoregulation leads to renal damage and subsequent failure. There are two mechanisms of autoregulation. First is the myogenic response, which is common to almost all microvascular beds. Second is tubuloglomerular feedback, which is a slower system that fine-tunes the delivery of solutes to each nephorn. Autoregulation is well studied. However, limitations in measurement techniques previously forced us to assume that all nephrons in the kidney act independently in response to local changes in blood pressure. The goal of this dissertation was to determine the presence, extent, and mechanism of network behaviour in renal autoregulation. 

In this dissertation I show that the mechanisms of renal autoregulation are synchronized over macroscopic regions of the renal surface, and that the spatial distribution of synchronization can be modulated by nitric oxide. Then I show that the patterns of synchronization of autoregulation on the surface of the kidney are governed by the underlying arterial anatomy. Finally, I show that the mechanism of internephron synchronization is chiefly dependent on gap junctional intercellular communication, and that removal of this communication pathway reduces the efficiency of autoregulation overall.

By the end, I have shown that internephron synchronization is present in the kidney, and 1) synchronization is modulated by an important hormonal modulator, nitric oxide, 2) that different arterial determinants can control what nephrons become synchronized, and 3) that degradation of synchronization impairs autoregulation. Overall, this shows that internephron synchronization is an important aspect of renal autoregulation, and it must be considered in the future when studying this field.

%\end{document}
