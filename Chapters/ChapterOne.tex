\chapter{Introduction}

\section{Background}
	\subsection{Kidney function}
		The primary function of the kidneys is to regulate the volume and composition of the extracellular fluid, while also excreting various waste products (\emph{e.g.} ammonia, excess water soluble vitamins) and synthesizing vitamin D\textsubscript{3} and erythropoetin \cite{GuytonHall}. Three main processes contribute to the maintenance of fluid and electrolyte balance: Filtration of plasma, reabsorption of solutes, and secretion of wastes.
		
	\subsection{Regulation of GFR}
		GFR is maintained at a high level, so that the kidneys can process the filtered plasma multiple times per day and perform rapid, fine adjustments to the extracellular fluid composition. Typically the kidneys receive approximately 20\% of cardiac output, even though they only account for 0.5\% of body mass. This disproportionately high blood supply places a heavy demand on the reabsorptive capacity of the kidney. For example, the kidneys normally filter approximately 180 L of plasma per day, but only 1.5 L are excreted in the urine per day. Therefore at least 99\% of the filtered plasma is reabsorbed each day. The importance of tight regulation of GFR is clear, because even a 5\% change in GFR from 180 L to 189 L per day would cause urine output to increase from 1.5 L to 10.5 L per day, if reabsorption were to remain constant. Given that a human's total blood volume is approximately 5 L, this type of change would be quickly lethal. The primary means of regulating GFR is mediated
		\subsubsection{Myogenic response}
			\emph{General myogenic response}\\
			
			 The myogenic response (MR) is the classic response of a blood vessel to constrict when its pressure is increased, and to dilate when pressure is decreased \cite{Bayliss02}. Increased pressure causes a non-selective cation channel to open in vascular smooth muscle cells \cite{Davis92}. Depolarization causes a downstream calcium entry \cite{Kotecha05}, which is likely from voltage-gated calcium channels \cite{Knot98}. Calcium influx increases Rho kinase activity, which phosphroylates the myosin regulatory light chain and increases its activity \cite{Schubert02}. This leads to contraction of each smooth muscle cell, and is dependent on actin polymerization \cite{Gokina02}. The transaction of pressure to opening of non selective cation channels is not clear \cite{Hill07}, but has been suggested to be transient receptor potential channel 6 (TRPC6) \cite{Welsh02} and TRPM4 \cite{Earley04}. 


\emph{Myogenic autoregulation in the kidney}\\
	
	Renal micro vessels respond to changes in pressure, and there is a preferential response of the preglomerular vasculature to pressure \cite{Gilmore80}. This notion was contested with the relatively new method of frequency domain analysis of renal blood flow \cite{Sakai86}, although later analyses showed that there is indeed a myogenic mechanism. Renal autoregulation was shown to have a working mechanism that operates at 0.1 - 0.3 Hz and is present in the hydronephrotic kidney, which lacks nephrons and only contains blood vessels \cite{Cupples98}. 
			
			
		\subsubsection{Tubuloglomerular feedback}
		
		Tubuloglomerular feedback (TGF) operates at 0.01-0.05 Hz \cite{Leyssac83,HolsteinRathlou86}. The general operation of TGF and the fact that it senses tubular sodium concentrations was originally demonstrated by Schermann \emph{et al.} (1970) \cite{Schnermann70}.
		
		\subsubsection{Other mechanisms}
			\emph{Third mechanism}\\
			The third mechanism is an unexplained slow oscillation.
			\emph{Fourth mechanism}\\
			The fourth mechanism seems to operate at the TGF frequency, but occurs even when TGF sensation is blocked with furosemide.
			\emph{Connecting tubule glomerular feedback}\\
			Connecting tubule glomerular feedback (CTGF) operates similarly to TGF, except that the sensors are epithelial sodium channels (ENaC) instead of NKCC.
	\subsection{Internephron interactions}
		\subsubsection{Oscillations in the kidney}
		\subsubsection{Synchronization of coupled oscillators}
	
		
		
			
 

\section{Conclusion}

\bibliographystyle{plainnat}
\renewcommand{\bibname}{References}
\bibliography{../Dissertation_Bib}
